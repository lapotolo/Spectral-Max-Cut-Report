\section{Spectral Max Cut}
We saw that the largest eigenvalue $ \lambda_n $ of the Laplacian matrix of a graph $ G $ is $ 2 $ if and only if $ G $ contains a bipartite connected component. \\
Since $ \lambda_n $ is a continuous value that has a discrete and combinatorial effect on the topology of the relative graph, we may hope to find a measure of how close a graph is to be bipartite with respect to how close is $ \lambda_n $ to $ 2 $. 
\\
%We have to introduce and show the correctedness and the soundness of some kind of parameter measuring this relations between the closeness of $ \lambda_n $ to $ 2 $ and how much a graph is close to be bipartite.
\\
We want a fractional parameter that is zero
if and only
the graph has a bipartite connected component, 
and that is close to zero if and only if the graph is
close to contain a bipartite connected components. 
\\
\\
We start the quest in defining this parameter by recalling the result we got for $ \lambda_n $ in the last section:
\[
\lambda_n = 2 - \min_{\substack{\mathbf{x} \in \mathbb{R}^n \\ \mathbf{x} \ne \mathbf{0}}} \frac{\sum_{(u,v)\in E}(x_u+x_v)^2}{d \sum_{v \in V} x_v^2 } \\
\iff 
2-\lambda_n = \min_{\substack{\mathbf{x} \in \mathbb{R}^n \\ \mathbf{x} \ne \mathbf{0}}} \frac{\sum_{(u,v)\in E}(x_u+x_v)^2}{d \sum_{v \in V} x_v^2 }  \]
Now we want to restrict this continuous minimization problem to the discrete case, that is we would like to move from looking for minimizer in $ \mathbb{R}^{|V|} $ to minimizer over a discrete field. 
\\
\\
The combinatorial problem equivalent to finding an almost bipartite connected component in a graph can be restated as: \\ \\
\emph{"given an unweighted, undirected, d-regular graph $ G = (V,E) $ look for a non-empty $ C \subseteq V $ and a bipartition $ (A,B) $ of $ C $ such that:
\\
\textbf{the ratio between the number of violating edges and incident edges to $ S $ is small.}
\\
We say that an edge $ (u,v)\, \in E $ is violating if it is an edge between two vertices in $ C $ and one endpoint is in $ A $ and the other in $ B $."}
\\
\\
The discrete field of our vector space that serves the purpose  is very close to the boolean one.\\
We need to distinguish between vertices in the cut $ C $ and vertices outside of it, so in $ V-C $.\\
Moreover we need to further distinguish between vertices that are in $ A \subseteq C $ and vertices in $ B \subseteq C $.\\
%We would like to have a vector over $ \mathbb{R}^{|V|} $ that assigns a value to each vertex of $ G $ with respect to the partition of $ G $ it is in.\\
Let $ \mathbf{y} $ be a vector in $ \{-1,0,1\}^{|V|} $ that maps vertices of $ G $ in the following manner: 
\begin{itemize}
\item $ v \in V $ is a node of $ A \subseteq C \iff y_v = -1 $;
\item $ v \in V $ is a node of $ B \subseteq C \iff y_v = 1 $;
\item $ v \in V $ is a node of $ V - C  \iff y_v = 0 $;
\end{itemize}
Keep in mind that $ A $ and $ B $ are a bipartition of $ C $. It means that $ A \cup B = C $ and $ A \cap B = \emptyset $. \\
We now define the \textbf{bipartiteness ratio of $ \mathbf{y} $} as
\[ \beta(\mathbf{y}) = \frac{\sum_{(u,v)\in E} |y_u+y_v|}{d \sum_{v \in V} y_v } \]
looking at how this parameter has been defined we can note that:
\begin{itemize}
\item the numerator counts for the two possible kinds of violating edges:
\begin{itemize}
\item edges contained in $ C $ that have endpoints both lying in $ A $ or both in $ B $ are counted with a weight of 2;
\item edges going from vertices in $ C $ to vertices in $ V-C $ are counted with a weight of 1;
\end{itemize}  
\item the denominator contains the sum of the degrees of the vertices of $ C $. For non regular graphs, this quantity is usually called the volume of $ C $ (indicated as $ vol(C) $).\\ Note that $ vol(V) = 2\cdot|E| $.
\end{itemize}
So in a more explicit way we can write
\[ \beta(\mathbf{y}) = \frac{2E(A,A) + 2E(B,B) + E(A \cup B, V - A \cup B)}{vol(A \cup B)} \]
Thus this parameter $ \beta(\mathbf{y}) $ stands for the fraction of edges of $ G $ incident to the cut $ C $, that should be removed to convert $ C $ into a bipartite connected component of $ G $.
\\ \\
We can now recast the expression of $ \beta(\mathbf{y}) $ into the combinatorial optimization problem equivalent to the continuous one regarding $ \lambda_n $.\\
We define the \textbf{bipartiteness ratio of the graph $ G $} as
\[ \beta(G) = \min_{\substack{\mathbf{y} \in \{-1,0,1\}^n \\ \mathbf{y} \ne \mathbf{0}}} \beta(\mathbf{y}) \]

Smaller values of $ \beta(G) $ corresponds to a smaller number of edges to remove from $ G $ to get a bipartite connected component in $ G $. 
\\
\\
It is left to find a robust relation connecting the discrete optimization problem that wants to compute $ \beta(G) $ and the continuos one that aims for $ \lambda_n $.\\
One of the so called Cheeger's inequalities formalizes this relation binding $ \lambda_n $ to the bipartiteness ratio of the graph $ \beta(G) $.

\begin{theorem}{\textbf{(Cheeger's inequalities for $ \lambda_n $, Trevisan)}}
\\
Let $ G $ be an undirected regular graph and let $ \lambda_1 \leq \lambda_2 \leq \dots \leq \lambda_n $ be the eigenvalues of the normalized laplacian, with repetitions, then
\[\frac{2-\lambda_n}{2} \leq \beta(G) \leq \sqrt{2 \cdot (2-\lambda_n)} \] 
\begin{proof}
the left part is the easy direction:
\begin{align*}
 2-\lambda_n & = \min_{\substack{\mathbf{x} \in \mathbb{R}^n \\ \mathbf{x} \ne \mathbf{0}}} \frac{\sum_{(u,v)\in E}(x_u+x_v)^2}{d \sum_{v \in V} x_v^2 } \\
 & \leq \min_{\substack{\mathbf{y} \in \{-1,0,1\}^n \\ \mathbf{y} \ne \mathbf{0}}} \frac{\sum_{(u,v)\in E} |y_u+y_v|^2}{d \sum_{v \in V} |y_v|^2 } \\
 & \leq \min_{\substack{\mathbf{y} \in \{-1,0,1\}^n \\ \mathbf{y} \ne \mathbf{0}}} \frac{\sum_{(u,v)\in E} 2 \cdot |y_u+y_v|}{d \sum_{v \in V} |y_v|} \\
 & = 2 \cdot \beta(G)
\end{align*}
\medskip

\textbf{In particular, proving this direction of the inequality is equivalent to state that if there exists a local bipartite connected component in $ G $ then $ \lambda_n $ is close to two} \\
The second inequality is harder to prove but it is useful to design a procedure that finds a cut $ C \subseteq V $ with a nice bipartiteness ratio.\\
The prove the other direction we need to apply the next lemma to an eigenvector $ \mathbf{x} $ of $ \lambda_n $.
\begin{lemma}[main]
For every $ \mathbf{x} \in \mathbb{R}^n, \,  \mathbf{x} \ne \mathbf{0} $ there is a threshold t \[ 0<t<\max_v |x_v| \] such that, if we define $ \mathbf{y}^{(t)} \in \{-1,0,1\}^n $ as
\begin{equation*}
\mathbf{y}_{v}^{(t)} = \begin{cases}
  -1 & \text{if } x_v \leq -t \\
  0 & \text{if }  -t \leq x_v < t \\
  1 & \text{if } x_v \geq t
 \end{cases}
\end{equation*}
we have
\[ \beta(\mathbf{y}^{(t)}) \leq \sqrt{2 \cdot \frac{\sum_{(u,v \in E)}(x_u+x_v)^2}{d \sum_{v \in V}x_v^2}} = \sqrt{2 \cdot (2-\lambda_n)}\]
\begin{proof}
look at \cite{trevi-1}, page 8.
\end{proof}
\end{lemma}
\end{proof}
\qed
\end{theorem}

Now it is possible to constructively use the second inequality
\[ \beta(\mathbf{y}^{(t)}) \leq  \sqrt{2 \cdot (2-\lambda_n)}\]
to give a procedure that returns a partition $ C $ of $ V $ yielding a nice bipartiteness ratio for $ G $.\\
The procedure does the following steps:
\begin{itemize}
\item sort the vertices according to $ |y_v| $;
\item process the vertices in the sorted order $ \{v_1, \dots, v_n \} $
\item for every cut $ C_k $ that is a suffix of the sorted order, bipartite $ C_k $ in $ (A,B) $ according to the sign of $ y_v $ and compute $ \beta(C_k) $;
\item return the cut $ C_k $ that minimizes its bipartiteness ratio;

\end{itemize}

We call this last procedure \textbf{QuasiBipartition} and by a simple combinatorial argument we can see that

\begin{proposition}{\textbf{QuasiBipartition terminates in polynomial time}} \\
Quasi-Bipartition returns a solution in time $ \mathcal{O}(|E| + |V| \log |V|) $
\begin{proof}
The term $  |V| \log |V| $ comes out from the initial sorting of the vertices according to their coordinates. \\
The cut that minimizes $ \beta(G) $ can be computed in $ \mathcal{O}(E) $, since the numerator of every $ \beta(C_k) $ is obtained iteratively reusing in part the numerator of the bipartiteness ratio of the previous step. \\
We start by computing the first $ \beta $ in constant time and in every following step we subtract edges that were violating for the previous cut and add other edges that are violating for the current cut. \\
Both the operations take time $ \mathcal{O}(d_{v_k}) $. 
All this operations take time  be done in $ \mathcal{O}(d_v) $ and thus repeating this argument for every vertex we get a total running time that is order $ \mathcal{O}(\sum_{v \in V} d_v) = \mathcal{O}(|E|) $ 
\qed
\end{proof}
\end{proposition}


\subsection{Non trivial approximation algorithm for MAX-CUT}
Finally we can give an approximation algorithm for MAX-CUT that exploits the procedure to compute $ \beta(G) $.\\
We call $ maxcut(G) $ the fraction of edges of $ G $ cut by an optimal solution $ S^* $. The link between this notation and the one used in the first section is
\[ maxcut(G) = \frac{COST(S^*)}{|E|}\]
We assume $ maxcut(G) = 1-\epsilon $, for any small $ \epsilon > 0 $, meaning that the optimal solution is not cutting all the edges of $ G $ and so $ G $ is not bipartite. 
\\
\\
Assume we have an optimal solution  $ S^* $ for MAX-CUT. \\
An algorithm that achieves a non trivial approximation ratio for MAX-CUT is based on this idea:
\begin{itemize}
	\item if $ COST(S^*) < (1-\epsilon) \cdot |E|$, then use a naive 0.5 approximation algorithm for MAX-CUT;
	\item if $ COST(S^*) \geq (1-\epsilon) \cdot |E| $ use the the procedure \textbf{QuasiBipartition} to compute $ \beta(G) $;
\end{itemize}
The branching choice is justified by the following fact
\begin{claim}
Let $S=V, A=S^*, B=V-S^*$ and suppose that
\[ COST(S^*) \geq (1-\epsilon) \cdot |E|\]
Then
\[ \beta(G) \leq \epsilon\]
\begin{proof}
\begin{align*}
\beta(G) \leq \beta(\mathbf{y})  &= \frac{2E(A,A)+2E(B,B)+E(V,V-S)}{vol(S)} \\
&= \frac{2E(A,A)+2E(B,B)}{vol(V)} \\
&= \frac{2(|E|- E(A, V-A))}{2|E|} \\
&\leq \frac{2(|E|- (1- \epsilon)|E|)}{2|E|}\\
&= \epsilon
\end{align*}
\qed
\end{proof}
\end{claim}


Recalling that by the left part of the Cheeger's inequalities for $ \lambda_n $ we have $\frac{2-\lambda_n}{2} \leq \beta(G) $, we derive $ 2-\lambda_n \leq 2 \epsilon $.\\ \\
Meaning that
\[ COST(S^*) \geq (1-\epsilon) \cdot |E| \iff 2-\lambda_n \leq \beta(G) \leq 2 \epsilon\]
And so know that it is safe to use the procedure \textbf{QuasiBipartition} since the possible values are bounded. \\
On the other hand it implies that 
\[ 2-\lambda_n > 2\epsilon \iff COST(S^*) < (1-\epsilon) \cdot |E| \]

So for example using the trivial randomized approximation algorithm we get
\[ \mathbb{E}[E(C,\bar{C})] = \frac{|E|}{2} = \frac{m}{2} \geq \frac{COST(S^*)}{2(1-\epsilon)} \]
Thus it is a $ \frac{1}{2(1-\epsilon)} $ approximation.

\subsection{Recursive Spectral Cut}
In case of graphs such that $ COST(S^*) \geq (1-\epsilon) \cdot |E| $, the idea is to exploit the procedure computing $ \beta(G) $ recursively on subgraphs of $ G $. The steps are:
\begin{itemize}
\item find $ \mathbf{y} $ that minimizes $ \beta(G) $;
\item remove the non-zero vertices from $ G $ (note that this is the safest choice since $ \mathbf{y} $ yields a way to partition them removing the minimum number of violating edges)
\item recur on $ G'=(V',E') $ where $ V' \subseteq V $ is the set of vertices such that $ \forall  v' \in V' : y_{v'} = 0 $
\item stop the recursion when the volume of the residual graph is very small.
\end{itemize}
Note that a small volume of the residual graph implies that we already have an approximated optimal global cut in the current residual. (think to the expression of $ \beta(\mathbf{y}) $)
\\
Otherwise if the volume is still large, the optimal cut of G is still a good cut in the next residual graph. This last statement implies that $ \lambda_n $ is close to 2 and that we can find a $ \mathbf{y'} $ that gives a small bipartiteness ratio.
\\ 
\\
More precisely the \textbf{RecursiveSpectralCut} overall algorithm is the following:
\begin{list}{$\bullet$}{}
	\item take in input $ G = (V,E) $;
	\item use \textbf{QuasiBipartition} on $ G $ to find the cut $ C $ and the bipartition $ (A,B) $ of $ C $ such that:
	\[ 2E(A,A) + 2E(B,B) + E(A \cup B, V - A \cup B) \leq \sqrt{2 \cdot (2-\lambda_n)} \cdot vol(A \cup B)  \]
	\item if $ V = A \cup B $
	\begin{itemize}
		\item return $ (A,B) $;
	\end{itemize}
	\item else
	\begin{itemize}
		\item let $ V' := V - (A \cup B) = V - C $ and let $ G' =(V',E')$ be the subgraph induced by $V'$;
		\item $ (C', V'-C') := $ \textbf{RecursiveSpectralCut}($ G' $)
		\item let $ A', B' $ be the two partitions of the cut  $ C' $  computed by the recursive call on $ G' $
		\item return the best cut between 
		$ (A' \cup A, B' \cup B)$ and $ (A' \cup B, B' \cup A)$
%$ (C \cup A, (V'- C) \cup B) $ and $ (C \cup B, (V'- C) \cup A) $
\end{itemize}
\end{list}
\bigskip

Trevisan in \cite{trevi-1} proved the following bound
\begin{lemma} if $ maxcut(G) = 1-\epsilon$, then RecursiveSpectralCut($ G $) finds a cut crossed by at least $(1-4\sqrt{\epsilon})\cdot |E| $ edges.
	\begin{proof}
		the proof is by induction on the number of recursive calls to RecursiveSpectralCut.
		\\
		\\
		\textbf{(base case)}
		\\
		\\
		In the base case the algorithm never recurs. \\
		So it must be the case that $ A \cup B = V $, and so $ (A,B) $ is already a cut of $ G $. \\
		Counting the number of edges of $ G $ not crossing the cut:
		\begin{align*}
		E(A,A) + E(B,B) + \overbrace{E(C, V-C)}^{0\,since\, V=C} & \leq \frac{1}{2} \sqrt{2 \cdot (2-\lambda_n)} \cdot vol(V) \\
& \leq 2\sqrt{\epsilon} \cdot 2|E| \\
& \leq 4 \sqrt{\epsilon} |E|
		\end{align*}

		since $ 2-\lambda_n \leq 2\epsilon $ and $ vol(V) = 2|E| $
		\\
		\\
		\textbf{(inductive step)}
		\\
		\\
		We start by assessing the number of uncut edges by the algorithm
		\[ (E(A,A) + E(B,B))
		+ \frac{1}{2}E(A \cup B, V') 
		+ \underbrace{(|E'| - E'(C,V'-C))}_\text{edges cut in the recursion} 
		\]
		because we have to count:
		\begin{enumerate}
			\item all the edges with both endpoints in $ A $ and both endpoints $ B $;
			\item half of the edges from $ A \cup B  $ to $ V' $.\\
			This is due to the fact that the best way of combining the cuts loses at most half of the edges. \\
			Infact every edge in $ E(A \cup B, V') $ either “stays on the same side”, going from A to $ A' $ or
			B to B' , or else “crosses sides”, going from A to $ B' $ or B to $ A' $ .\\
			That means that one of the above cuts must contain at least 1/2 the edges in $ E(A \cup B, V') $;
			\item all the edges of $ G' $ not cut in the recursive step;
		\end{enumerate}
		by using $ 2-\lambda_n \leq 2\epsilon $ it follows
		\[  E(A,A) + E(B,B) + \frac{1}{2}E(A \cup B, V') \leq \sqrt{\epsilon} \cdot vol(A \cup B) \]
		and by the inductive hypothesis
		\[ |E'| - E'(C,V'-C) \leq 4\sqrt{\epsilon'} \cdot |E'| \]
		where $ \epsilon' $ is such that $ maxcut(G') = 1-\epsilon' $ \\\\
		Call the fraction of edges of $ G $ not in $ G' $
		\[ \rho\ = \frac{|E|-|E'|}{|E|}\]
		Then we have:
		\[vol(A \cup B) \leq 2|E-E'| = 2\rho |E| \]
		and
		\[ |E'| = |E| \cdot (1-\rho) \]
		Recalling that the number of edges not cut by the optimal cut of $ G' $ is at most the number of edges not cut by the optimal cut of $ G $, given that $ G' $ is a subgraph of $ G $, follows that
		\[ \epsilon'|E'| \leq \epsilon|E| \implies \epsilon' \leq \epsilon \cdot \frac{|E|}{|E|} = \epsilon \cdot \frac{1}{1- \rho} \]
		The total number of edges not cut by the algorithm is at most
		
		\[ \sqrt{\epsilon} \cdot vol(A \cup B) + 4\sqrt{\epsilon'} \cdot |E'| \]
		\[ \leq \sqrt{\epsilon} \cdot 2\rho|E| + 4\sqrt{\epsilon \cdot (1 - \rho)} |E| \]
		\[\leq 4\sqrt{\epsilon} |E| \]
		where in the last line we used
		\[  4\sqrt{\epsilon}|E| =  2 \rho + 4 \sqrt{1- \rho} \leq 4 \]
		which follows from
		\[ \sqrt{1- \rho} \leq \sqrt{(1- \frac{\rho}{2})^2} = 1- \frac{\rho}{2} \]
		
		We can conclude that the total number of cut edges is at most 
		\[ |E| - \#\text{uncut edges} = |E|- 4\sqrt{\epsilon}|E| = (1 - 4\sqrt{\epsilon})|E|  \]
		\qed
	\end{proof}
\end{lemma}

Going a bit looser with the bounds we get a 0.529 approximation algorithm for MAX-CUT. We follow a similar argument assuming the same procedure explained above.
\\
\\
Assume we are in the case $ COST(S^*) \geq (1-\epsilon)\cdot |E|$.\\
We use \textbf{RecursiveSpectralCut} to compute an approximate solution $ S $ for MAX-CUT on the graph $ G $.

\[ COST(S) \geq E(A,B) + \frac{1}{2}E(C,V-C) + COST(S-C) \]
where $ COST(S-C) $, with some abuse of notation, is the cost of MAX-CUT on the residual graph $ G'=(V',E') $.

By some simple combinatorial calculations we upper bound the optimal solution with
\begin{align*}
COST(S^*) &\leq E(A,A) + E(B,B) + E(A,B) + E(C,V-C) + COST(S^*-C)\\
& = \frac{1}{2} vol(C) + \frac{1}{2}E(C,V-C) + COST(S^*-C)
\end{align*}
Then 
\begin{align*}
\frac{COST(S)}{COST(S^*)} & \geq \min \bigg\{ \frac{E(A,B)}{\frac{1}{2}vol(C)}
	                              , \frac{\frac{1}{2}E(C,V-C)}{\frac{1}{2}E(C,V-C)}
	                              , \frac{COST(S-C)}{COST(S^*-C)}\bigg\}\\
&=\min \bigg\{ \frac{E(A,B)}{\frac{1}{2}vol(C)}
             , \frac{COST(S-C)}{COST(S^*-C)}\bigg\}
\end{align*}
Using the hypothesis $2-\lambda_n \leq 2\epsilon $ and the fact
\[ \frac{2E(A,A) + 2E(B,B) + E(C, V - C)}{vol(A \cup B)} = \frac{vol(C)-2E(A,B)}{vol(C)} \]
\\
\\
we bound $ \frac{E(A,B)}{\frac{1}{2}vol(C)} $ by
\[ 2 \sqrt{\epsilon} \geq \frac{vol(C)-2E(A,B)}{vol(C)} = 1 - \frac{2E(A,B)}{vol(C)} \implies \frac{E(A,B)}{\frac{1}{2}vol(C)} \geq 1-2\sqrt{\epsilon} \]
\\
\\
While to bound $  \frac{COST(S-C)}{COST(S^*-C)} $ it is enough to note that the same reasoning must hold true for the residual graph $ G-C $.\\
However we should take into account that at some point during the recursion residual graphs may have $ 2-\lambda_n \geq 2\epsilon $. In these cases the approximation ratio is $ \frac{1}{2(1-\epsilon)} $ since the procedure uses a trivial approximation algorithm.
\\
\\
Putting the pieces together we get
\[\frac{COST(S)}{COST(S^*)} \geq \min \bigg\{ 1-2\sqrt{\epsilon}, \frac{1}{2(1-\epsilon)} \bigg\}\]
The two expressions are equal for $ \epsilon \approx 0.0554 $. At that point the ratio is 0.529. So this procedures yields at least a 0.529 approximation.


\subsection{Conclusions}
\textbf{RecursiveSpectralCut} on its own works well in case of graphs that are almost bipartite, that is when $ \epsilon $ is small, since this would imply that $ maxcut(G) $ is very close to $ 1 $.
\\
\\
Unfortunately in case of large $ \epsilon $ the spectral algorithms does not find nice cuts. \\
Infact if $ \epsilon > \frac{1}{64} $, the algorithm guarantees to cut less then half the edges of the graph, and this is worse than the performances of the three trivial approximation algorithms we saw.
\\
\\
Trevisan proved that by always choosing the best cut between the one returned by \textbf{RecursiveSpectralCut} and the one returned by the greedy algorithm, we can deal with large $ \epsilon $ as well, achieving an approximation ratio equal to 0.531. That is indeed better than the trivial one. Goemans\cite{goe} in 1995 gave a 0.878-approximation algorithm for MAX CUT by using semidefinite programming. Even if SDP approaches yielded better approximation ratio so far, it may be fruitful to further investigate spectral approaches. Infact computing eigenvectors is lot easier both in terms of speed and memory than solving semidefinite programming; experimental results indicate that this algorithm performs better than the SDP-based algorithm both in time spent and result quality.
\\
\\
Finally both spectral graph theory and spectral MAX-CUT are hot research fields. It is not known if there is a "one-shot" spectral algorithm not using recursive calls. (recursion makes it hard to analyze the algorithm, and forces recomputation of eigenvectors.)\\
The bound proved in \cite{trevi-1} is not the tightest. In a recent research \cite{soto-improv}, Soto proved that the same procedure reaches an approximation ratio up to 0.614247.
