\section{Spectral Max Cut}
We saw that the largest eigenvalue $ \lambda_n $ of the Laplacian matrix of a graph $ G $ is $ 2 $ if and only if $ G $ contains a bipartite connected component. Since $ \lambda_n $ is a continuous value that has a discrete and combinatorial effect on the topology of the relative graph, we may hope to find a measure of how close a graph is to be bipartite with respect to how close is $ \lambda_n $ to $ 2 $. \\
We have to introduce and show the correctedness of and the soundness of a parameter.
We want this fractional parameter to be zero
if and only
the graph has a bipartite connected component, 
and to be small if and only if the graph is
close to contain a bipartite connected components. 
\\
\\
We start the quest in defining this parameter by recalling the result we got for $ \lambda_n $ in the last section:
\[
\lambda_n = 2 - \min_{\substack{\mathbf{x} \in \mathbb{R}^n \\ \mathbf{x} \ne \mathbf{0}}} \frac{\sum_{(u,v)\in E}(x_u+x_v)^2}{d \sum_{v \in V} x_v^2 } \\
\iff 
2-\lambda_n = \min_{\substack{\mathbf{x} \in \mathbb{R}^n \\ \mathbf{x} \ne \mathbf{0}}} \frac{\sum_{(u,v)\in E}(x_u+x_v)^2}{d \sum_{v \in V} x_v^2 }  \]
Now we want to restrict this continuous minimization problem to the discrete case, that is we would like to move from looking for minimizer in $ \mathbb{R}^{|V|} $ to minimizer over a discrete field. 
\\
\\
The combinatorial problem equivalent to finding an almost bipartite connected component in a graph can be restated as: \\ \\
\emph{"given an unweighted, undirected, d-regular graph $ G = (V,E) $ look for a non-empty $ C \subseteq V $ and a bipartition $ (A,B) $ of $ C $ such that:
\\
\textbf{the ratio between the number of violating edges and incident edges to $ S $ is small.}
\\
We say that an edge $ (u,v)\, \in E $ is violating if it is an edge between two vertices in $ C $ and one endpoint is in $ A $ and the other in $ B $."}
\\
\\
The discrete field of our vector space that serves the purpose  is very close to the boolean one.\\
We need to distinguish between vertices in the cut $ C $ and vertices outside of it, so in $ V-C $.\\
Moreover we need to further distinguish between vertices that are in $ A \subseteq C $ and vertices in $ B \subseteq C $.\\
%We would like to have a vector over $ \mathbb{R}^{|V|} $ that assigns a value to each vertex of $ G $ with respect to the partition of $ G $ it is in.\\
Let $ \mathbf{y} $ be a vector in $ \{-1,0,1\}^{|V|} $ that maps vertices of $ G $ in this manner: 
\begin{itemize}
\item $ v \in V $ is a node of $ A \subseteq C \iff y_v = -1 $;
\item $ v \in V $ is a node of $ B \subseteq C \iff y_v = 1 $;
\item $ v \in V $ is a node of $ V - C  \iff y_v = 0 $;
\end{itemize}
Keep in mind that $ A $ and $ B $ are a bipartition of $ C $. It means that $ A \cup B = C $ and $ A \cap B = \emptyset $. \\
We now define the \textbf{bipartiteness ratio of $ \mathbf{y} $} as
\[ \beta(\mathbf{y}) = \frac{\sum_{(u,v)\in E} |y_u+y_v|}{d \sum_{v \in V} y_v }  \]
looking at how this parameter has been defined we can note that:
\begin{itemize}
\item the numerator counts for the two possible kinds of violating edges: edges contained in $ C $ that have endpoints both lying in $ A $ or both in $ B $ are counted with a weight of 2; edges going from vertices in $ C $ to vertices in $ V-C $ are counted with a weight of 1.
\item the denominator contains the sum of the degrees of the vertices of $ C $. For non regular graphs, this quantity is usually called the volume of $ C $ (indicated as $ vol(C) $).\\ In case of regular graph $ vol(C) \leq 2\cdot|E| $.
\end{itemize}

We can also define the \textbf{bipartiteness ratio of the graph $ G $} as
\[ \beta(G) = \min_{\substack{\mathbf{y} \in \{-1,0,1\}^n \\ \mathbf{y} \ne \mathbf{0}}} \beta(\mathbf{y}) \]

This parameter $ \beta(G) $ can be interpreted as the fraction of edges of $ G $ incident to $ C $ that have to be removed to convert $ C $ into a bipartite connected component of $ G $.\\
Smaller values of $ \beta(G) $ corresponds to a smaller number of edges to remove from $ G $ to get a local bipartite connected component in $ G $. 
\\
\\
%We will see that graphs with $ \lambda_n $ close to $ 2 $ are graphs that locally have a connected component that is close to a bipartite one.
One of the so called Cheeger's inequalities formalizes this relation and binds $ \lambda_n $ to the bipartiteness ratio of the graph.
%To formalize what has been just said about $ \beta(G) $, we will prove one of the so called Cheeger's inequalities that binds $ \lambda_n $ to this bipartiteness ratio of the graph.
\begin{theorem}{\textbf{(Cheeger's inequalities for $ \lambda_n $, Trevisan)}}
\\
Let $ G $ be an undirected regular graph and let $ \lambda_1 \leq \lambda_2 \leq \dots \leq \lambda_n $ the eigenvalues of the normalized laplacian, with repetitions, then
\[\frac{2-\lambda_n}{2} \leq \beta(G) \leq \sqrt{2 \cdot (2-\lambda_n)} \] 
\begin{proof}
the left part is the easy direction:
\begin{align*}
 2-\lambda_n & = \min_{\substack{\mathbf{x} \in \mathbb{R}^n \\ \mathbf{x} \ne \mathbf{0}}} \frac{\sum_{(u,v)\in E}(x_u+x_v)^2}{d \sum_{v \in V} x_v^2 } \\
 & \leq \min_{\substack{\mathbf{y} \in \{-1,0,1\}^n \\ \mathbf{y} \ne \mathbf{0}}} \frac{\sum_{(u,v)\in E} |y_u+y_v|^2}{d \sum_{v \in V} |y_v|^2 } \\
 & \leq \min_{\substack{\mathbf{y} \in \{-1,0,1\}^n \\ \mathbf{y} \ne \mathbf{0}}} \frac{\sum_{(u,v)\in E} 2 \cdot |y_u+y_v|}{d \sum_{v \in V} |y_v|} \\
 & = 2 \cdot \beta(G)
\end{align*}
\medskip

\textbf{In particular, proving this direction of the inequality is equivalent to state that if there exists a local bipartite connected component in $ G $ then $ \lambda_n $ is close to two} \\
The second inequality is hard to prove but it is useful to design an algorithm that finds a cut $ C \subseteq V $ with a nice bipartiteness ratio.\\
The prove the other direction we need to apply the next lemma to an eigenvector $ \mathbf{x} $ of $ \lambda_n $.
\begin{lemma}[main]
For every $ \mathbf{x} \in \mathbb{R}^n, \,  \mathbf{x} \ne \mathbf{0} $ there is a threshold t \[ 0<t<\max_v |x_v| \] such that, if we define $ \mathbf{y}^{(t)} \in \{-1,0,1\}^n $ as
\begin{equation*}
\mathbf{y}_{v}^{(t)} = \begin{cases}
  -1 & \text{if } x_v \leq -t \\
  0 & \text{if }  -t \leq x_v < t \\
  1 & \text{if } x_v \geq t
 \end{cases}
\end{equation*}
we have
\[ \beta(\mathbf{y}^{(t)}) \leq \sqrt{2 \cdot \frac{\sum_{(u,v \in E)}(x_u+x_v)^2}{d \sum_{v \in V}x_v^2}} = \sqrt{2 \cdot (2-\lambda_n)}\]
\begin{proof}
look at \cite{trevi-notes}, page 30-31.
\end{proof}
\end{lemma}
\end{proof}
\qed
\end{theorem}

Now it is possible to constructively use the second inequality
\[ \beta(\mathbf{y}^{(t)}) \leq  \sqrt{2 \cdot (2-\lambda_n)}\]
to give a procedure that returns a partition $ C $ of $ V $ yielding a nice bipartiteness ratio for $ G $.\\
The procedure does the following steps:
\begin{itemize}
\item sort the vertices according to $ |y_v| $;
\item process the vertices in the sorted order $ \{v_1, \dots, v_n \} $
\item for every cut $ C_k $ that is a suffix of the sorted order, bipartite $ C_k $ in $ (A,B) $ according to the sign of $ y_v $ and compute $ \beta(C_k) $;
\item return the cut $ C_k $ that minimizes its bipartiteness ratio;

\end{itemize}


Following a simple combinatorial argument we can see that
\begin{proposition}{\textbf{Quasi-Bipartition terminates in polynomial time}} \\
Quasi-Bipartition returns a solution in time $ \mathcal{O}(|E| + |V| \log |V|) $
\begin{proof}
The term $  |V| \log |V| $ comes out from the initial sorting of the vertices according to their coordinates. \\
The cut that minimizes $ \beta(G) $ can be computed in $ \mathcal{O}(E) $, since the numerator of every $ \beta(C_k) $ is obtained iteratively reusing in part the numerator of the bipartiteness ratio of the previous step. \\
We start by computing the first $ \beta $ in constant time and in every following step we subtract edges that were violating for the previous cut and add other edges that are violating for the current cut. \\
Both the operations take time $ \mathcal{O}(d_{v_k}) $. 
All this operations take time  be done in $ \mathcal{O}(d_v) $ and thus repeating this argument for every vertex we get a total running time that is order $ \mathcal{O}(\sum_{v \in V} d_v) = \mathcal{O}(|E|) $ 
\qed
\end{proof}
\end{proposition}


\subsection{Spectral MAX-CUT}
Finally we can give an approximation algorithm for MAX-CUT that exploits $ \beta(G) $.\\
The idea is to apply the procedure to compute $ \beta(G) $ recursively:
\begin{itemize}
\item find $ \mathbf{y} $ that minimizes $ \beta(G) $;
\item remove the non-zero vertices from $ G $ (note that this is the safest choice since $ \mathbf{y} $ yields a way to partition them removing the minimum number of violating edges)
\item recur of $ G'=(V',E') $ where $ V' \subseteq V $ is the set of vertices such that for each $ v' $ $y_{v'} = 0 $
\item stop the recursion when the volume of the residual graph is very small.
\end{itemize}
Note that when the volume of the residual is small it implies that we already have a global cut.\\
Otherwise if the volume is still large, the optimal cut of G is still a good cut in the residual graph. This last statement implies that $ \lambda_n $ is close to 2 and that we can find a $ \mathbf{y'} $ that gives a small bipartiteness ratio.