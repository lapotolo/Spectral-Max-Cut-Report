\section{Introduction: Graphs, cuts and hardness of approximation}

\subsection{MAX-CUT Problem}
Graphs are a mathematical construction typically used to model connections and relations between things. \\
Sticking to the undirected case, a graph is a pair of sets $G=(V,E)$ where $V$ represents the collection of things to be connected, commonly called vertices, while $E$ is a set of unordered pairs of elements of $V$, modelling the relation that exists between the objects of our universe.

Given an undirected graph $G = (V,E)$, a cut in $G$ is a partition of the vertices of $ G $ graph into two disjoint subsets $ C,\bar{C} \subseteq V$ such that $\bar{C} = V \setminus C $, $ C \cap \bar{C} = \emptyset$ and $ C \cup \bar{C} = V $.
Let $E(C,\bar{C})$ denote the set of edges with one vertex in $C$ and one vertex in $\bar{C}$. \\
The MAX-CUT problem consists in finding a cut $C$ that maximizes $|E(C,\bar{C})|$, namely the number of edges crossing the cut. \\
A weighted version of MAX-CUT also exists and there things get more complicated.	
In these instances we are also given a function $w : E \rightarrow R$ that assigns a weight to every edge, so that the problem is to find a cut with maximum weight. The MAX-CUT problem is NP-hard.
It is interesting to recall the contrast with the MIN-CUT problem, which admits a linear time solutions by using a randomized algorithm.

\subsection{Approximation Algorithms}
%A PTAS is an algorithm which takes an instance of an optimization problem and a parameter $ \epsilon > 0  $ and, in polynomial time, produces a solution that is within a factor $ 1 + \epsilon  $of being optimal (or $ 1 − \epsilon  $ for maximization problems)
MAX-CUT as many other real life problems are optimization problems.
In the standard complexity theory formulation, problems are given in a decisional form, meaning that what is asked is the existence of at least one element in the space of solutions.
In an optimization framework the task is to build the best solution among all the feasible ones.
A large number of these optimization problems are known to be in NP-HARD, that is no polynomial time, or space, algorithm computing an exact solution is known, and probably it did not even exist.\\

However their existence needs to be taken into account since lines of attack serve to deal with their complexity and make them more tractable.
One possible plan is to exploit the notion of approximate solution.
Approximate algorithm theory is the branch of complexity theory that let formalize the notion of approximate solution for optimization problems.
More precisely we have:
\begin{itemize}
	\item $ \Pi $, an optimization problem whose solving algorithm is in NP-HARD,
	\item $ S^{*} $, an optimal solution for the instance $ I $ of $ \Pi $,
	\item $ S $, an approximate solution for the instance $ I $ of $ \Pi $,
	\item a function $ COST: \mathcal{S}(I_{\Pi}) \rightarrow \mathbb{N} $, where $ \mathcal{S}(I_{\Pi})$ is the space of solution of the optimization problem $ \Pi $ with instance $ I $. ($ \mathcal{S}(I_{\Pi}) \rightarrow \mathbb{R} $ in the continuous case)
\end{itemize}
The goal becomes finding a procedure that given and instance $ I $ of $ \Pi $ computes an approximation solution $ S $ such that:
\begin{align*}
	\frac{COST(S)}{COST(S^*)} \leq r && \text{if we have a minimization problem} \\
	\frac{COST(S^*)}{COST(S)} \leq r && \text{if we have a maximization problem}
\end{align*}

In both cases a lower value for $ r $ implies a better algorithm.\\
In general is difficult to estimate the value of $ COST(S^{*}) $. So as we will see in the proves in the next sections we will need to use an upperbound to $ COST(S^{*}) $ in case of 
maximization problems (or a lower bound in case of minimization problems). 
So by letting $ UB $ be a solution yielding a cost that is an upperbound of the optimal solution we have:
\[ COST(S) \leq COST(S^{*}) \leq COST(UB) \]
And we will prove the theorems stating the value of $ r $ by showing:
\[ 
\frac{COST(UB)}{COST(S)} \leq r 
\Longleftrightarrow
COST(S) \geq \frac{COST(UB)}{r} \]